\section{向量空间}

\subsection{复数}

省略...
\subsection{向量空间的定义}
若我们定义平面,它由所有有序实数对构成
\begin{equation}
    \textbf{R}^{2} =\left\{ \left ( x,y \right ):x,y\in \textbf{R}\right\}
\end{equation}
定义空间:我们定义为
\begin{equation}
    \textbf{R}^{2} =\left\{ \left ( x,y,z\right ):x,y,z\in \textbf{R}\right\}
\end{equation}
如果定义高纬度 便引出组的概念
组(list)
\begin{enumerate}
    \item 按顺序排列
    \item 用逗号隔开,用括弧(),()括弧里面可以是任何东西
    \item 特别的 其中长度大于零,可以用角标来表示第几个
\end{enumerate}

\begin{quote}
    区分与集合的区别 :
    \begin{enumerate}
        \item 对于组  (有序)\\
        例如:$\left ( 0,0,0,0 \right )\neq \left ( 0 \right )$
        \item 对于集合(无序) \\
        例如:$\left\{0,0,0,0 \right\}=\left\{ 0\right\}$
    \end{enumerate}
\end{quote}
\par 其中对于数域:\\
\begin{center}
    $F^n \quad R+C$ :\quad R-Real \quad C- Complex 
\end{center}

{\bfseries 记忆简化}
\begin{equation}
    \mathbf{F^{n}}=\left\{ (x_{1},...(x_{n}):x_{j}\in F,j=1,...,n\right\}\\
    \to x_{1},...,x_{n},y_{1},...,y_{n} \in\textbf{F}\\
\end{equation}

\begin{equation}
    \left ( x_{1},...,x_{n} \right )+\left (y_{1},...,y_{n}  \right )=\left ( x_{1}+y_{1},...,x_{n}+y_{n} \right )\\
    \to x+y=y+x
\end{equation}
\newline
\par 其中用单个字母表示$F^n$中的一个元素,那么必须列出坐标时,通常要标有坐标角标来表示,
比如,若$x\in F^n$,则令x等于$(x_{1},…,x_{n})$就是很好的表示方法:例如0的表示:\\
\begin{equation}
    0=\left ( 0,...,0 \right )
\end{equation}

\newpage

\subsection{向量的加法与乘法}
{\bfseries 向量的加减法和标量乘法\\}


其中性质如下
\begin{itemize}
    \item 交换性\\
对于所有的 ,$\textbf{u},\textbf{v}\in V $,都有,$\textbf{u}+\textbf{v}=\textbf{v}+\textbf{u}$
    \item 结合性
    \item 加法单位元\\
存在一个元素$ 0\in V$,使得对所有的,$v\in V$, 都有, $v+0=v$;
    \item 加法逆\\
对于每一个,$v\in V$,都存在 $w\in V$, 使得,$w+v=0$
    \item 乘法单位元
    \item 分配性质\\
对于$a,b \in \textbf{F}(数),\textbf{u},\textbf{v} \in V$,都有 ,$a(\textbf{u}+\textbf{v})=a\textbf{u}+a\textbf{v};(a+b)\textbf{u}=a\textbf{u}+b\textbf{u}$
\end{itemize}

\subsection{向量空间的性质}
\begin{enumerate}
    \item 向量空间有唯一的加法单位元
    \item 向量空间中的每一个元素都有唯一的加法逆
    \item 对于每一个 $\textbf{v}\in V$ ,都有$0\textbf{v}=\textbf{0}$
    \item 对于每一个$a\in F$,都有$a\textbf{0}=\textbf{0}$
    \item 对于每一个$\textbf{v}\in V$,都有$(-1)\textbf{v}=-\textbf{v}$
\end{enumerate}


\subsection{子空间}
定义:\\
V 的子集 U称为V的$\textbf{子空间(subspace)}$,如果U(采用V相同的加法和标量乘法)也是向量空间
满足条件:
\begin{itemize}
    \item 加法单位元\\
$\textbf{0}\in U$
    \item 加法封闭\\
$\textbf{u+v}\in U$
    \item 标量乘法封闭\\
$若a\in F,\textbf{u}\in U,则 a\textbf{u}\in U$
\end{itemize}

\subsection{和与直和}
定义:
设 $U_{1},...,U_{m}$ 都是$V$的子空间,则$U_{1},...,U_{m}$的和记作$U_{1}+...+U_{m}$,
定义为$U_{1},...,U_{m}$中所有可能的和\\
\begin{equation}
    U_{1}+...+=U_{m}=\left\{u_{1}+...+u_{m}:u_{1}\in U_{1},... ,u_{m}\in U_{m}\right\}
\end{equation}
\par 使得$V=U_{1}+...+U_{m}$,使得 V 中每个元素都可以写成如下形式

$$
u_{1}+...+u_{m}
$$

称为V是子空间$U_{1},...,U_{m}$的直和,记$V=U_{1}\oplus...\oplus U_{m}$\\
定理:\\
\begin{center}
    设$U_{1},...,U_{m}$都是V的子空间,则$V=U_{1}\oplus...\oplus U_{m}$
\end{center}
当且仅当下两个条件成立:
\begin{enumerate}
    \item $V=U_{1}+...+ U_{m}$
    \item 若 $\textbf{0}=u_{1}+...,u_{n},u_{j}\in U_{j}$\\
设U和W都是V的空间,则$V=U\oplus...\oplus W $当且仅当$V=U+W$,并且$U\cap W=\left\{ 0\right\}$
\end{enumerate}