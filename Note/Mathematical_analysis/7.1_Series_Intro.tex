% !TEX root = main.tex

\section{无穷级数与广义积分}
\label{sec:series}
本章中会将正项级数、一般项级数、广义积分(无穷限积分、瑕积分)、函数项级数放在一起进行比较,当然也会涉及各类级数各自的特性. 由于将内容进行整合,故最重要的各类判别法总结在\ref{summary_conver}节才进行讲述.

\subsection{级数的基本性质}
\begin{definition}[无穷级数]
$\{u_n\}$为数列,称形式和
\[\ssum{u_n}=\lim_{n\to\infty}\sum_{k=1}^n u_k=\lim_{n\to\infty}S_n=S\]
为级数$\ssum{u_n}$的和,称其收敛到$S$.
\end{definition}
\par 可由数列极限性质直接推导出来的定理在此不再赘述.
\begin{theorem}[数项级数收敛的必要条件]
\label{series_conver}
若$\ssum{u_n}$收敛,则$\displaystyle\lim_{n\to\infty}u_n=0$
\end{theorem}
\par 注意此定理并非充分条件,如$\ssum{\ln\lrp{1+\dfrac{1}{n}}}=\ln(n+1)$发散,但其一般项趋于$0$. 并且,此定理也只适用于数项级数,广义积分并不具有这样的性质.
\par 在进行数项级数收敛性判断时,第一步就得先判定其一般项是否趋于$0$,然后再考虑其他判别方法. 如果遇到比较简单的数列,可以直接求和的,那就先求和再讨论,如下面的例\ref{eg_series_conver1}.
\begin{example}
$\ssum{r^n\sin nx},|r|<1$
\label{eg_series_conver1}
\end{example}
\begin{analysis}
用三角函数的复数形式. 由棣莫弗公式$e^{ix}=\cos x+i\sin x$,得到$\sin x=\dfrac{e^{ix}-e^{-ix}}{2}$,其中$i^2=-1$.
\[\begin{aligned}
LHS &= \dfrac{1}{2}\ssum{r^n(e^{inx}-e^{-inx})}\\
&= \dfrac{1}{2}\lim_{n\to\infty}\left(re^{ix}\dfrac{1-(re^{ix})^n}{1-re^{ix}}-re^{-ix}\dfrac{1-(re^{-ix})^n}{1-re^{-ix}}\right)\qquad\mbox{等比数列求和}\\
&\xlongequal{z=re^{ix}} \dfrac{1}{2}\left(\dfrac{z}{1-z}-\dfrac{\bar{z}}{1-\bar{z}}\right)\qquad\mbox{因$|z|<1$故$\displaystyle\lim_{n\to\infty}z^n=0$}\\
&= \dfrac{1}{2}\dfrac{z-\bar{z}}{1-(z+\bar{z})+r^2}\qquad\mbox{$z\bar{z}=r^2$,通分}\\
&= \dfrac{r\sin x}{1-2r\cos x+r^2}
\end{aligned}\]
\par 注:当然也可以用三角函数积化和差和差化积去做,但相比起复数法会复杂很多
\end{analysis}
关于数列求和的各种技巧,如裂项求和、错位相消、分子有理化、三角恒等变形等均要熟悉.


\subsubsection{正项级数}
\begin{definition}[正项级数]
级数的每一项都\textbf{非负},则称其为正项级数. 由此知正项级数一定为实级数.
\end{definition}
\begin{theorem}[正项级数收敛的充要条件]
正项级数$\ssum{u_n}$收敛的充要条件是$\{S_n\}$有上界
\end{theorem}
\begin{analysis}
	必要性:极限有界性;充分性:单调有界原理
\end{analysis}
\begin{example}p级数
\[\ssum{\dfrac{1}{n^p}}\]
当$p\leq 1$时发散,$p>1$时收敛($p=1$时也称为调和级数)
\end{example}
\par 下面的柯西积分判别法是用连续量来研究离散量很好的例子,遇到$\ln n,\dfrac{1}{n^p}$等可考虑
\begin{theorem}[柯西积分判别法]
$f(x)$在$[0,+\infty)$连续且单调下降,$u_n=f(n)$,则$\ssum{u_n}$收敛充要条件是
\[\displaystyle\lim_{x\to+\infty}\int_1^xf(t)\diff t\]
存在
\end{theorem}
\begin{example}
\[\sum_{n=2}^{\infty}{\dfrac{1}{n^p\ln n}},p\in\rr\]
\end{example}
\begin{analysis}
\begin{enumerate}
	\item 当$p=1$时,$f(x)=\dfrac{1}{x\ln x}$在$[2,+\infty)$单调递减
	\[\limtoinf\intab{2}{n}{\dfrac{1}{x\ln x}}=\ln\ln x\Big|_2^{+\infty}=+\infty\]
	由柯西积分判别法,原级数发散
	\item 当$p\leq 1$时,$\dfrac{1}{n^p\ln n}\geq\dfrac{1}{n\ln n}$,由比较判别法,原级数发散
	\item 当$p>1$时,$\dfrac{1}{n^p\ln n}<\dfrac{1}{n^p}$,由比较判别法,原级数收敛
\end{enumerate}
\end{analysis}

\subsubsection{一般项级数}
\begin{definition}[交错级数]
$\ssum{(-1)^{n-1}u_n},u_n>0$
\end{definition}
\par 下面的莱布尼茨判别法是判断交错级数收敛简便而有效的方法.
\begin{theorem}[莱布尼茨(Leibniz)判别法]
交错级数$\ssum{(-1)^{n-1}u_n}$的一般项$\{u_n\}$单调下降趋于$0$,则交错级数收敛
\end{theorem}
\par 注意莱布尼茨判别法是判断收敛性,而不是判断绝对收敛性的,后面多种判别法一起运用时经常会搞混. 由于前面的有限项对最终的极限没有影响,因此交错级数的一般项只要从某个$n$开始单调下降趋于$0$就可以使用莱布尼茨判别法了.
\par 当然不是所有交错级数都能让你一眼望穿,直接使用判别法,如下面这个例子就需要进行拆分,再分别判断.
\begin{example}
\[\sum_{n=2}^{\infty}\dfrac{(-1)^n}{\sqrt{n}+(-1)^n}\]
\end{example}
\begin{analysis}泰勒展开
\[\dfrac{(-1)^n}{\sqrt{n}+(-1)^n}=\dfrac{(-1)^n}{\sqrt{n}}\lrp{1+\dfrac{(-1)^n}{\sqrt{n}}}^{-1}
=\dfrac{(-1)^n}{\sqrt{n}}\left[1-\dfrac{(-1)^n}{\sqrt{n}}+\mathcal{O}\lrp{\dfrac{1}{n}}\right]
=\dfrac{(-1)^n}{\sqrt{n}}-\dfrac{1}{n}+\mathcal{O}\lrp{\dfrac{1}{n^{\frac{3}{2}}}}\]
因$\nums{\dfrac{1}{\sqrt{n}}}$单调下降趋于$0$,由莱布尼茨判别法,$\displaystyle\sum_{n=2}^{\infty}\dfrac{(-1)^n}{\sqrt{n}}$收敛\\
又由$p$级数的性质知$\displaystyle\sum_{n=2}^{\infty}\dfrac{1}{n^{\frac{3}{2}}}$收敛.\\
但$\displaystyle\sum_{n=2}^\infty\dfrac{1}{n}$发散,故原级数发散.
\end{analysis}
\par 下面的阿贝尔变换和阿贝尔引理均为阿贝尔判定法的铺垫,证明均不难,记住几何意义可以现推.
\begin{theorem}[阿贝尔(Abel)变换]
两组数列$\{a_n\},\{b_n\}$,设$\{b_n\}$的部分和数列为$\{B_n\}$,则
\[\begin{aligned}
\sum_{i=1}^na_ib_i&=\sum_{i=1}^na_i(B_i-B_{i+1})\\
&=\sum_{i=1}^{n-1}(a_i-a_{i+1})B_i+a_nB_n
\end{aligned}\]
\end{theorem}
\begin{theorem}[阿贝尔引理]
设$\{a_n\}$单调,$\{B_n\}$为$\{b_n\}$的部分和有界$M$,则
\[\left|\sum_{i=1}^na_ib_i\right|\leq M(|a_1|+2|a_n|)\]
\end{theorem}

\subsubsection{代数运算}
\label{sec:subsub:series_operation}
\begin{enumerate}
	\item 结合律:\textbf{条件/绝对收敛}级数任意加括号,和不变;但不能随意去括号
	\item 交换律:\textbf{绝对收敛}成立;条件收敛适当重排,可使新级数发散或收敛到某一特定值(Riemman)
	\item 分配律:\textbf{绝对收敛}成立(Cauchy)
\end{enumerate}
\par 下面对级数的乘法再做深入讨论.
\par 设$\ssum{a_n}=A,\ssum{b_n}=B$,考虑乘积$\lrp{\ssum{a_n}}\lrp{\ssum{b_n}}$,可以发现有多种不同的求和顺序,如
\begin{itemize}
	\item 对角线法(柯西乘积)
\begin{center}
\begin{tikzcd}
a_1b_1 & a_1b_2\arrow{ld} & a_1b_3\arrow{ld} & \cdots\arrow{ld} & a_1b_m\\
a_2b_1 & a_2b_2\arrow{ld} & a_2b_3\arrow{ld} & \cdots & a_2b_m\\
a_3b_1 & a_3b_2\arrow{ld} & a_3b_3 & \cdots & a_3b_m\\
\vdots & \vdots & \vdots & \ddots & \vdots\\
a_nb_1 & a_nb_2 & a_nb_3 & \cdots & a_nb_m
\end{tikzcd}
\end{center}
\[S=a_1b_1+(a_1b_2+a_2b_1)+(a_1b_3+a_2b_2+a_3b_1)+\cdots=\sum_{k=2}^\infty\sum_{i=1}^{k-1}a_ib_{k-i}\lrp{=\sum_{k=0}^\infty\sum_{i=0}^ka_ib_{k-i}\quad\mbox{(从$0$开始)}}\]
	\item 矩形法
\begin{center}
\begin{tikzcd}
a_1b_1 & a_1b_2\arrow{d} & a_1b_3\arrow{d} & \cdots & a_1b_m\\
a_2b_1 & a_2b_2\arrow{l} & a_2b_3\arrow{d} & \cdots & a_2b_m\\
a_3b_1 & a_3b_2\arrow{l} & a_3b_3\arrow{l} & \cdots & a_3b_m\\
\vdots & \vdots & \vdots & \ddots & \vdots\\
a_nb_1 & a_nb_2\arrow{l} & a_nb_3\arrow{l} & \cdots & a_nb_m
\end{tikzcd}
\end{center}
\[\begin{aligned}
S&=a_1b_1+(a_1b_2+a_2b_2+a_2b_1)+(a_1b_3+a_2b_3+a_3b_3+a_3b_2+a_3b_1)+\cdots\\
&=\sum_{k=1}^\infty\lrp{\sum_{i=1}^{k}a_ib_k+\sum_{i=1}^{k-1}{a_kb_{k-i}}}
\end{aligned}\]
\end{itemize}
\par 关于相乘的结果,有以下结论.
\begin{theorem}[级数相乘]
分以下三种情况讨论
\begin{enumerate}
	\item (Cauchy) 若$A$和$B$都绝对收敛,则以任意方式相乘,结果都为$AB$
	\item (Mertens) 若$A$或$B$中有一者条件收敛,另外一者绝对收敛,则柯西乘积为$AB$
	\item 若$A$和$B$都条件收敛,则最终结果不一定收敛
\end{enumerate}
\end{theorem}


\subsection{广义积分}
\begin{definition}[无穷限积分]
设$f(x)$在$[a,+\infty)$有定义,并且在任意有限区间$[a,A]$上可积,定义
\[\int_a^{+\infty}f(x)\diff x:=\lim_{A\to+\infty}\int_a^Af(x)\diff x=I\]
为$[a,+\infty)$的无穷限积分
\end{definition}
\begin{definition}[瑕积分]
设$f(x)$在$(a,b]$有定义,并且在任意区间$[a+\eta,b]$上可积($\eta>0$),在$(a,a+\eta]$无界,定义
\[\intab{a}{b}{f(x)}:=\lim_{\eta\to0^+}\intab{a+\eta}{b}{f(x)}\]
为$[a,+\infty)$的瑕积分
\end{definition}
\par 注意在无穷项积分中$\intab{a}{+\infty}{f(x)}$收敛并不能推出$f(x)\to0(x\to+\infty)$,这与数项级数不同,如例\ref{countereg_conver_integ}说明了这一点.
\par 下面的积分第二中值定理类似于数列的阿贝尔变换,用于证明广义积分的阿贝尔判定法.
\begin{theorem}[积分第二中值定理]
设$f(x)$在$[a,b]$上可积,而$g(x)$在$[a,b]$上单调,则在$[a,b]$中存在$\xi$使得
\[\intab{a}{b}{f(x)g(x)}=g(a)\intab{a}{\xi}{f(x)}+g(b)\intab{\xi}{b}{f(x)}\]
\end{theorem}
\par 在进行广义积分的收敛性判定时,一般先尝试其是否可以进行正常的积分,即求不定积分(详细方法见第\ref{sec:indefinite_integration}章),之后才考虑用各类判别法. 并且在实际运算过程中要记得分段,最好每一个区间只含一个瑕点,在端点或者内点都可.
\begin{example}
\[\intab{0}{+\infty}{\dfrac{\sqrt{x}}{1+x^2}}\]
\end{example}
\begin{analysis}
如果只是判断收敛性,则用比较判别法
\[\lim_{x\to\infty}\dfrac{\sqrt{x}}{1+x^2}\Big/\dfrac{1}{x^{\frac{3}{2}}}=\lim_{x\to\infty}\dfrac{x^2}{1+x^2}=1\]
而$\intab{0^+}{+\infty}{\dfrac{1}{x^{\frac{3}{2}}}}$收敛(注意瑕点),故原积分收敛.\\
如果要求值,则变得相当麻烦,法一:
\[\begin{aligned}
  \intab{0}{+\infty}{\dfrac{\sqrt{x}}{1+x^2}}
  &=\intabu{0}{+\infty}{\dfrac{2u^2}{1+u^4}}{u}\\
  &=\intabu{0}{+\infty}{\left(-\dfrac{\sqrt{2}/2u}{u^2+\sqrt{2}u+1}+\dfrac{\sqrt{2}/2u}{u^2-\sqrt{2}u+1}\right)}{u}\\
  &=\dfrac{\sqrt{2}}{2}\Bigg(\dfrac{1}{2}\int_0^{+\infty}\dfrac{\diff(u^2-\sqrt{2}u+1)}{u^2-\sqrt{2}u+1}+\dfrac{\sqrt{2}}{2}\int_0^{+\infty}\dfrac{\diff\left(u-\sqrt{2}/{2}\right)}{\left(u-\sqrt{2}/{2}\right)^2+\left(\sqrt{2}/{2}\right)^2}\\
  &\quad-\dfrac{1}{2}\int_0^{+\infty}\dfrac{\diff(u^2+\sqrt{2}u+1)}{u^2+\sqrt{2}u+1}+\dfrac{\sqrt{2}}{2}\int_0^{+\infty}\dfrac{\diff\left(u+\sqrt{2}/{2}\right)}{\left(u+\sqrt{2}/{2}\right)^2+\left(\sqrt{2}/{2}\right)^2}\Bigg)\\
  &=\dfrac{\sqrt{2}}{2}\left(\dfrac{1}{2}\ln\dfrac{u^2-\sqrt{2}u+1}{u^2+\sqrt{2}u+1}+\dfrac{\sqrt{2}}{2}\arctan(\sqrt{2}u-1)+\dfrac{\sqrt{2}}{2}\arctan(\sqrt{2}u+1)\right)\Bigg|_0^{+\infty}\\
  &=\dfrac{\sqrt{2}}{2}\pi                   
\end{aligned}\]
法二:倒代换,令$u=\dfrac{1}{t}$,则原式变为$\disp -2\intabu{0}{+\infty}{\dfrac{1}{1+t^4}}{t}$,接下来步骤同例\ref{egfrac1}.
\end{analysis}

\subsection{函数项级数}
\subsubsection{函数列}
对于函数列$\{f_n(x)\}$可以将其看为二元函数$f(n,x)$,但实际上我们在研究它的性质的时候往往是针对某个点$x=x_0$(确定的$x$),令$n\to\infty$. 这样看的话,函数项级数与数项级数没有太大差别.
\begin{definition}[极限函数]
\[f(x):=\lim_{n\to\infty}f_n(x),\forall x\in X\]
称为函数列$\{f_n(x)\}$的极限函数
\end{definition}
由于在不同点函数值构成的数列收敛的快慢不一致,在快慢发生急剧变化的地方,极限函数可能会变成间断,因此引入\textbf{一致收敛}的概念,以描述函数各处收敛速度差不多的性质.
\begin{definition}[一致收敛]
$\forall \varepsilon>0,\exists N=N(\varepsilon),\forall n>N, x\in X$,有
\[|f_n(x)-f(x)|<\varepsilon\]
则称函数列$\{f_n(x)\}$在$X$一致收敛到$f(x)$. 或记
\[\rho_n=\sup_{a\leq x\leq b}|f_n(x)-f(x)|\]
$\{f_n(x)\}$在$X$一致收敛到$f(x)$当且仅当$\rho_n\to0(n\to\infty)$.
\end{definition}
\par 函数列的一致收敛性一般按照定义证明,与极限定义证明方法类似,在此不再赘述,关键在于把$x$看成常数,$n$看成变量.
\par 注意不一致收敛的定义与平时的定义改写有较大区别,在此说明.
\begin{definition}[不一致收敛]
$\exists \varepsilon_0>0,\forall N\in\zz^+,\exists n>N$与$x_n\in[a,b]$,使得
\[|f_n(x_n)-f(x)|\geq\varepsilon_0\]
则称函数列$\{f_n(x)\}$在$X$不一致收敛到$f(x)$
\end{definition}
\par 下面是一个关于一致收敛性很有趣的例子.
\begin{example}
\label{ex:interesting_uniform_convergence}
\[f_n(x)=\dfrac{x^n}{1+x^n}\]
讨论在以下区间的一致收敛性
\begin{enumerate}
	\item $x\in[0,b],b<1$
	\item $x\in[0,1)$
\end{enumerate}
\end{example}
\begin{analysis}易得极限函数
\[f(x)=\begin{cases}
0&x\in[0,1)\\
\dfrac{1}{2}&x=1\\
1&x>1
\end{cases}\]
\begin{enumerate}
	\item $\forall\varepsilon>0,|f_n(x)-f(x)|=\dfrac{x^n}{1+x^n}\leq x^n\leq b^n,x\in[0,b],b<1$\\
	则取$N=\lrfloor{\log_b\varepsilon}+1$,当$n>N$时,$|f_n(x)-f(x)|\leq \varepsilon$,故一致收敛
	\item $\exists\varepsilon_0=\dfrac{1}{3},\forall N,\exists n=N+1>N$,取$x_n=\dfrac{1}{\sqrt[n]{2}}\in[0,1)$,则
	\[|f_n(x_n)-f(x_n)|=\dfrac{1/2}{1+1/2}=\dfrac{1}{3}\geq\eps_0=\dfrac{1}{3}\]
	故不一致收敛
\end{enumerate}
初看会感觉这两者很矛盾,但细想是有区别的. 前者的上确界小于$1$,而后者的上确界等于$1$,其中的极限过程已经超出了人脑能脑补的程度了.
\end{analysis}
\begin{definition}[一致有界]
\[\exists M>0, s.t. \forall x\in X,n:\;|f_n(x)|\leq M\]
则称函数列$\{f_n(x)\}$在$X$一致有界
\end{definition}
\begin{theorem}
\begin{enumerate}
	\item $f_n(x)$在$[a,b]$上有界,并且$\nums{f_n(x)}$在$[a,b]$上一致收敛到$f(x)$,则$f_n(x)$在$[a,b]$上一致有界
	\item $f_n(x)$在$[a,b]$上一致连续,并且$\nums{f_n(x)}$在$[a,b]$上一致收敛到$f(x)$,则$f(x)$在$[a,b]$上一致连续
	\item $f_n(x)$在$[a,b]$上黎曼可积,并且$\nums{f_n(x)}$在$[a,b]$上一致收敛到$f(x)$,则$f(x)$在$[a,b]$上黎曼可积
\end{enumerate}
\end{theorem}
\begin{analysis}
\begin{enumerate}
	\item 因$f_n(x)$在$[a,b]$有界,设$|f_n(x)|<M_n$\\
	因$f_n(x)\to f(x)$,取$\eps=1$,则当$n>N$时,$|f_n(x)-f(x)|<1$,故$|f(x)|<|f_{N+1}(x)|+1<M_{N+1}+1$\\
	进而$\forall n>N:|f_n(x)|<|f(x)|+1<M_{N+1}+2$,取$M=\max(M_1,M_2,\ldots,M_N,M_{N+1}+2)>0$,则$\forall n:|f_n(x)|<M$,$f_n(x)$在$[a,b]$上一致有界
	\item 最后要证$|f(x_1)-f_{N+1}(x_1)+f_{N+1}(x_1)-f_{N+1}(x_2)+f_{N+1}(x_2)-f(x_2)|<\eps$,拆分一下分别用定义即可
	\item 用定积分定义证,比较麻烦
\end{enumerate}
\end{analysis}
\begin{theorem}[函数列的分析性质]
\begin{enumerate}
	\item (连续)若函数列$\{f_n(x)\}$每一项都在$[a,b]$连续,且$\{f_n(x)\}$在$[a,b]$一致收敛到$f(x)$,则$f(x)$连续
	\[\displaystyle\lim_{x\to x_0}\lim_{n\to\infty}f_n(x)=f(x_0)=\lim_{n\to\infty}\lim_{x\to x_0}f_n(x)\]
	\item (可积)若函数列$\{f_n(x)\}$每一项都在$[a,b]$连续,且$\{f_n(x)\}$在$[a,b]$一致收敛到$f(x)$,则$f(x)$可积
	\[\intab{a}{b}{f(x)}=\displaystyle\lim_{n\to+\infty}\intab{a}{b}{f_n(x)}\]
	\item (可微)若函数列$\{f_n(x)\}$每一项都在$[a,b]$有连续微商$f_n'(x)$,$\{f_n(x)\}$在$[a,b]$\textbf{逐点}收敛到$f(x)$,\\
	且$\{f_n'(x)\}$在$[a,b]$\textbf{一致}收敛到$\sigma(x)$,则$f(x)$可微,且
	\[\displaystyle f'(x)=(\lim_{n\to+\infty}f_n(x))'=\lim_{n\to+\infty}f_n'(x)=\sigma(x)\]
\end{enumerate}
\end{theorem}
\begin{example}
\begin{enumerate}
	\item 若函数列$\{f_n(x)\}$每一项都在$[a,b]$连续,且$\{f_n(x)\}$在$[a,b]$一致收敛到$f(x)$,$x_n\in[a,b]$,且$\limtoinf{x_n}=x_0$,则$\limtoinf{f_n(x_n)}=f(x_0)$
	\item $\{f_n(x)\}$在$[a,b]$一致收敛到$f(x)$,$x_0\in(a,b)$且$\disp\lim_{x\to x_0}f_n(x)=a_n$,则$\disp\limtoinf\lim_{x\to x_0}f_n(x)=\lim_{x\to x_0}\limtoinf{f_n(x)}$
\end{enumerate}
\end{example}
\begin{analysis}
\begin{enumerate}
	\item 证$|f_n(x_n)-f(x_0)|=|f_n(x_n)-f(x_n)+f(x_n)-f(x_0)|<\eps$
	\item 证$|f(x)-a|=|f(x)-f_{N+1}(x)+f_{N+1}(x)-a_{N+1}+a_{N+1}-a|<\eps$,其中会用到柯西列
\end{enumerate}
\end{analysis}

\subsubsection{和函数}
\begin{definition}[和函数]
对于函数列$\nums{u_n(x)}$,定义和函数
\[\begin{aligned}
S(x):&=\lim_{n\to\infty}S_n(x)=\lim_{n\to\infty}\sum_{k=1}^nu_k(x)\\
&=\ssumk{u_k(x)},\forall x\in X
\end{aligned}\]
\end{definition}
\par 当函数项级数可以求和时,要先把和求出来. 把$x$看成常数,函数项级数会往往变成一个等比数列,这样就可以求和了.
\begin{example}按定义讨论下面级数的一致收敛性
\[\ssum{\dfrac{(-1)^{n-1}x^2}{(1+x^2)^n}},x\in(-\infty,+\infty)\]
\end{example}
\begin{analysis}
设部分和数列为$S_n(x)$,和函数(若存在的话)为$S(x)$,则$\forall\eps>0,\exists N=N(\eps)$使得$\forall n>N$有
\[|S_n(x)-S(x)|<\eps\]
注意到
\[S_n(x)=\ssumkn{\dfrac{(-1)^{k-1}x^2}{(1+x^2)^k}}=-x^2\ssumkn{\lrp{-\dfrac{1}{1+x^2}}^k}=\dfrac{x^2}{2+x^2}\lrp{1-\lrp{-\dfrac{1}{1+x^2}}^n}\]
故
\[\limtoinf{S_n(x)}=S(x)=\dfrac{x^2}{2+x^2}\]
因此
\[\begin{aligned}
|S_n(x)-S(x)|
&=\lrabs{\dfrac{x^2}{2+x^2}\dfrac{(-1)^{n-1}}{(1+x^2)^n}}\\
&=\dfrac{x^2}{(1+x^2)^n(2+x^2)}\\
&\leq\dfrac{1}{2}\dfrac{x^2}{1+nx^2}\qquad\mbox{伯努利不等式}\\
&\leq\dfrac{1}{2}\dfrac{x^2}{nx^2}\\
&=\dfrac{1}{2n}<\eps
\end{aligned}\]
取$N=\lrfloor{\dfrac{1}{2\eps}}+1$,则当$n>N$时上式成立,故一致收敛
\end{analysis}
\par 其实函数项级数就和普通的数项级数没有太大差别,用在数项级数的判别法,在函数项级数这里照样可以用,需要改动的地方不过是在一些地方加上“一致”罢了. 下面这条定理就是定理\ref{series_conver}的函数项级数版本.
\begin{theorem}[函数项级数一致收敛必要条件]
函数项级数$X$一致收敛的必要条件是一般项构成的函数序列$\{u_n(x)\}$在$X$一致收敛于$0$
\end{theorem}
\par 而下面这个判别法则是数项级数没有的判别法,非常强大,屡用不爽.
\begin{theorem}[M判别法(Weierstrass)]
若对函数项级数$\ssumk{u_k(x)}$,存在$M_k$使得
\[|u_k(x)|\leq M_k,\forall x\in X\]
而正项级数$\ssumk{M_k}$收敛,则$\ssumk{u_k(x)}$在$X$一致收敛
\end{theorem}
\par 技巧在于,把一般项里含$x$的项都放缩去掉,只留下关于$n$的项
\begin{example}
若函数列$\{u_n(x)\}$的每一项都是$[a,b]$的单调函数,$\ssum{u_n(x)}$在$[a,b]$的端点处绝对收敛,则该级数在$[a,b]$上一致收敛
\end{example}
\begin{analysis}由函数的单调性,及M判别法
\[u_n(x)<\max(|u_n(a)|,|u_n(b)|)\]
故$\{u_n(x)\}$在$[a,b]$上一致收敛
\end{analysis}
\par 当$x$实在消不掉时,可采用一些“巧妙”的方法,如下.
\begin{example}
\[\ssum{\dfrac{nx}{1+n^4x^2}},|x|>0\]
\end{example}
\begin{analysis}
因直接对分母用均值不等式,得到$\dfrac{1}{n}$并无法判断收敛性,故换种方式\\
$\forall x_0,|x_0|>0,\exists\delta>0,\delta\in(0,|x_0|),\forall|x|\geq\delta$,有
\[\lrabs{\dfrac{nx}{1+n^4x^2}}\leq\dfrac{n|x|}{n^4x^2}=\dfrac{1}{n^3|x|}\leq\dfrac{1}{n^3\delta}\]
而$\ssum{\dfrac{1}{n^3\delta}}$收敛,由M判别法,原级数在$|x|\geq\delta$一致收敛\\
又$x_0$的任意性,知原级数在$|x|>0$一致收敛
\end{analysis}
\begin{theorem}[迪尼(Dini)]
若函数列$\{u_n(x)\}$在$[a,b]$连续,$u_n(x)\geq 0$,函数项级数$\ssum{u_n(x)}$在$[a,b]$\textbf{逐点}收敛到$S(x)$,$S(x)$在$[a,b]$连续,则$\ssum{u_n(x)}$在$[a,b]$一致收敛
\end{theorem}
\begin{theorem}[和函数的分析性质]
\label{thm:sum_function_analysis_properties}
与一般函数列的分析性质非常类似
\begin{enumerate}
	\item (连续)若函数列$\{u_n(x)\}$在$[a,b]$连续,且函数项级数$\ssum{u_n(x)}$在$[a,b]$一致收敛到$S(x)$,则$S(x)$在$[a,b]$连续
	\item (逐项积分)若函数列$\{u_n(x)\}$在$[a,b]$连续,且函数项级数$\ssum{u_n(x)}$在$[a,b]$一致收敛到$S(x)$,则$S(x)$可积
	\[\intab{a}{b}{S(x)}=\ssum{\intab{a}{b}{u_n(x)}}\]
	\item (逐项微分)若函数列$\{u_n(x)\}$在$[a,b]$有连续微商$u_n'(x)$,$\ssum{u_n(x)}$在$[a,b]$\textbf{逐点}收敛到$S(x)$,\\
	且$\ssum{f_n'(x)}$在$[a,b]$\textbf{一致}收敛到$\sigma(x)$,则$S(x)$可微,且
	\[\displaystyle S'(x)=\left(\ssum{u_n(x)}\right)'=\ssum{u_n'(x)}=\sigma(x)\]
\end{enumerate}
\end{theorem}
\begin{example}
若函数列$\{u_n(x)\}$在$[a,b]$连续,且函数项级数$\ssum{u_n(x)}$在$(a,b)$一致收敛,则$\ssum{u_n(x)}$在$[a,b]$一致收敛
\end{example}
\begin{analysis}
取一个小于$\eps$的值,对$\ssum{u_n(x)}$用柯西收敛原理后,两侧取极限$x\to a^+,x\to b^-$,即得证. (因连续性取极限会导致取等,故要取小于$\eps$的值)
\end{analysis}
\par 关于和函数分析性质的运用,可以见下例.
\begin{example}
\label{eg:sum_fun_analysis}
\[\intab{-\pi}{\pi}{\dfrac{1-r^2}{1-2r\cos x+r^2}}=2\pi,|r|<1\]
\end{example}
\begin{analysis}
由例\ref{eg_series_conver1}知
\[\dfrac{1-r^2}{1-2r\cos x+r^2}=1+2\ssum{r^n\cos nx},|r|<1\]
当$|r|<1$时,$|2r^n\cos nx|\leq 2|r|^n$,而$\ssum{2|r|^n}$收敛,由M判别法知$\ssum{r^n\cos nx}$一致收敛\\
又级数的每一项$\{r^n\cos nx\}$都在$(-\infty,+\infty)$连续,故可逐项积分,得
\[\intab{-\pi}{\pi}{\dfrac{1-r^2}{1-2r\cos x+r^2}}=\intab{-\pi}{\pi}{}+2\ssum{\intab{-\pi}{\pi}{r^n\cos nx}}=2\pi\]
\end{analysis}