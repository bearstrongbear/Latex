% !TEX root = main.tex

\section{场论初步}
\label{sec:field}
\[\mathbf{F}(x,y,z)=(P(x,y,z),Q(x,y,z),R(x,y,z))=P\vi+Q\vj+R\vk\]
梯度场
\[\grad u=\nabla u=\lrp{\pd{}{x},\pd{}{y},\pd{}{z}}u=\lrp{\pd{u}{x},\pd{u}{y},\pd{u}{z}}\]
散度场
\[\opdiv \mathbf{F}(x,y,z)=\opdiv\cdot\vF=\pd{P}{x}+\pd{Q}{y}+\pd{R}{z}\]
旋度场\footnote{叉积的计算方式如下
\[\vx\times\vy=\vmat{\vi&\vj&\vk\\x_1&x_2&x_3\\y_1&y_2&y_3}=\vmat{x_2&x_3\\y_2&y_3}\vi+\vmat{x_1&x_3\\y_1&y_3}\vj+\vmat{x_1&x_2\\y_1&y_2}\vk\]}
\[\rot \mathbf{F}(x,y,z)=\nabla\times\vF=\vmat{\vi&\vj&\vk\\\pd{}{x}&\pd{}{y}&\pd{}{z}\\P&Q&R}\]
高斯公式可改写为
\[\iiint_V(\opdiv\vF)\diff V=\oiint_\mS\vF\cdot\diff\vS\]
斯托克斯公式可改写为
\[\iint_\mS\rot\vF\cdot\diff\vS=\oint_\mL\vF\cdot\diff\vl\]
\begin{definition}
若对$V$内任一逐段光滑曲线$L$
\begin{enumerate}
	\item 且$L$封闭,则$\disp\oint_L\vF\cdot\diff\vs=0$,则$\vF$为有势场
	\item $\disp\int_L\vF\cdot\diff\vs$与路径无关,则$\vF$为保守场
\end{enumerate}
\end{definition}
\begin{theorem}
设$V$为空间的单连通区域,$\vF=(P,Q,R)$在$V$内有连续偏导数,则$\vF$是有势场、保守场、梯度场、无旋场等价
\end{theorem}