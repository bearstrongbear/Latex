% !TEX root = main.tex

\section{微商与微分}
\subsection{微分}
课本上关于微分和微商的定义并不自然,不能让读者理解到其背后深刻的内涵,因此在这里我们尝试换个角度来理解.
\begin{definition}
对于函数$f(x)$,存在线性函数$L(x)$满足
\[\lim_{x\to x_0}\frac{|f(x)-L(x)|}{x-x_0}=0\]
则称$L(x)$是$f(x)$的线性逼近(linear approximation).
对应地,如果存在$H(x)$满足
\[\lim_{x\to x_0}\frac{|f(x)-H(x)|}{(x-x_0)^n}=0\]
则称$H(x)$是$f(x)$的高阶($n$阶)逼近(high order approximation).
\end{definition}
\begin{definition}[可微性]
\label{differentiability}
函数$f(x)$在$x_0$可微,当且仅当它在$x=x_0$存在一个线性逼近
\end{definition}
上述定义告诉我们,若$f(x)$在$x_0$可微,则其线性逼近$L(x)$靠近$f(x_0)$的速度比$x$靠近$x_0$的速度快;而所谓的高阶逼近\footnote{高阶可微意味着存在高阶逼近,但高阶可微不代表高阶可导},则对应着泰勒展开(\ref{sec:sub:taylor}节);更特殊地,我们可以定义出$0$阶逼近,那它代表的就是$f(x)$在$x=x_0$连续了. 因而我们也就能够理解下面这条熟知的定理了
\begin{theorem}
可导一定连续,连续不一定可导
\end{theorem}
因为可导(存在线性/一阶逼近)是比连续(存在常数/$0$阶逼近)更加强的性质,故可导是蕴含着连续的.
\par 更一般地,结合线性代数的知识,我们可以有以下定义
\begin{definition}
\label{def:multi_derivative_intro}
对于函数$f:\mathbb{R}^n\to\mathbb{R}^m$,如果存在线性变换$J(\mathbf{x_0}):\mathbb{R}^n\to\mathbb{R}^m$使得
\[\lim_{\mathbf{h}\to \mathbf{0}}\frac{\|f(\mathbf{x}+\mathbf{h})-(f(\mathbf{x_0})+\mathbf{J}(\mathbf{x_0})\cdot \mathbf{h})\|_{\mathbb{R}^n}}{\|\mathbf{h}\|_{\mathbb{R}^m}}=0\]
则称$f(\mathbf{x})$在$\mathbf{x_0}$可微,线性变换$T(\mathbf{x_0})$称为$f$在$\mathbf{x_0}$处的导数.
\par 特殊地,取$n=m=1$,$f(x)$为一元函数;取$m=1$,$f(x)$为多元函数.
\end{definition}
实际上这个定义也是采用逼近的思想,用一个$n$维的线性映射来逼近一个$n$元的函数.\footnote{多元函数中可微与可导的直观区别是什么? - 余翔的回答 - 知乎 \url{https://www.zhihu.com/question/23468713/answer/35850882}}
\par 结合定义\ref{differentiability}我们可以得到,对于一元函数,可微与可导等价,因为$f(x)$可微就存在一个线性逼近$L(x)=J(x_0)(x-x_0)+f(x_0)$,也就对应着这样一个线性变换$T(x_0)$,即其导数,故可导;反过来,$f(x)$可导,即存在一个线性变换$J(x_0)$,也就存在一个线性逼近$L(x)$同上定义符合逼近性质,进而推出$f(x)$可微.
\par 但对于多元函数,可微(可以全微分)说明存在一个与曲面相切的(超)平面,那其各个方向的偏导数都应存在;而可导(可求偏导)只是存在某个方向的切线,不能推出可微.
\par 至于高阶微分只需知道以下一些事实:
\begin{enumerate}
	\item 一阶微分具有形式不变性(即$\diff y=f'(u)\diff u,\diff y=\varphi'(x)\diff x$形式相同),而高阶微分没有(即$\diff^ny=f^{(n)}(x)\diff x^n$不成立,乘法法则会多出几项),故高阶微分不可随意进行变量代换
	\item 高阶微商的莱布尼茨公式
	\[y^{(n)}=\sum_{k=0}^nC_n^ku^{(n-k)}v{(k)}\]
	\item 高阶微分的表示\footnote{为什么二阶导数要这么记? - Milo Yip的回答 - 知乎 \url{https://www.zhihu.com/question/23166546/answer/23797756}}:
	\[\ddx(\ddx(y))=(\ddx\ddx)y=(\ddx)^2y=\ddxs y=\dfrac{\diff^2 y}{\diff x^2}\]
	故$\diff x^2=(\diff x)^2$,而$\diff^2 y$自己作为一个算子整体,不可拆分
\end{enumerate}

\subsection{求导}
链式法则、对数求导法、隐函数求导、参数方程微分法等等在这里均不作详细说明,将各种求导公式(特别是三角反三角)记熟即可.

\subsection{微分中值定理}
\begin{theorem}[费马(Fermart)定理]
$f(x)$在$\mathring{U}(x_0)$有定义,在$x_0$达到极值,且$f(x)$在$x_0$可导,则$f'(x_0)=0$
\end{theorem}
\begin{theorem}[罗尔(Rolle)定理]
$f(x)$在$[a,b]$连续,在$(a,b)$可导,$f(a)=f(b)$,则$\exists\xi\in(a,b)$,$f'(\xi)=0$
\end{theorem}
\begin{theorem}[拉格朗日(Lagrange)中值定理(微分中值定理)]
$f(x)$在$[a,b]$连续,在$(a,b)$可导,则$\exists\xi\in(a,b)$,使得
\[f'(\xi)=\frac{f(b)-f(a)}{b-a}\]
或者
\[\theta=\frac{\xi-a}{b-a}\in(0,1),\xi=a+\theta(b-a),f(b)-f(a)=f'(a+\theta(b-a))(b-a)\]
\end{theorem}
\begin{theorem}[柯西(Cauchy)中值定理]
$f(x),g(x)$在$[a,b]$连续,在$(a,b)$可导,$g'(x)\ne 0$,则$\exists\xi\in(a,b)$,使得
\[\frac{f'(\xi)}{g'(\xi)}=\frac{f(b)-f(a)}{g(b)-g(a)}\]
\end{theorem}
\begin{theorem}
$f(x),g(x),h(x)$在$[a,b]$连续,在$(a,b)$可导,则$\exists\xi\in(a,b)$,使得
\begin{equation*}
\begin{vmatrix}
f(a) & g(a) & h(a)\\
f(b) & g(b) & h(b)\\
f'(\xi) & g'(\xi) & h'(\xi)
\end{vmatrix}=0
\end{equation*}
\end{theorem}
需要理清以下几个关系:\textbf{连续可导(continuously differentiable)}\footnote{\url{https://math.stackexchange.com/questions/1117323/the-definition-of-continuously-differentiable-functions}}是指\textbf{导函数连续},不是指原函数连续并且可导,因为可导已经蕴含了(可推出)连续,所以没必要再强调;也不是指二阶可导,由于连续不一定可导;但\textbf{二阶可导}蕴含了连续可导.
\begin{definition}[光滑函数]
若$f(x)$连续,则称其为$C^0$函数;若$f(x)$连续可导,则称其为$C^1$函数;若$f(x)$满足$n$阶连续可导,则称其为$C^n$函数. $\forall n,f(x)\in C^n$,则称$f(x)$光滑,并称其为$C^\infty$函数
\end{definition}
\begin{example}
\label{egcondiff}
\[f(x)=\begin{cases}
\displaystyle x^m\sin\frac{1}{x},\quad x\ne 0\\
0,\quad x=0\end{cases}
\quad(m\in\mathbb{Z}^+)\]
\end{example}
\begin{analysis}
\begin{enumerate}
	\item $m\geq 1,f(x)$在$x=0$连续
	\[\lim_{x\to 0}f(x)=\lim x^m\sin\frac{1}{x}=f(0)=0\]
	要满足上式且在$x<0$部分有定义,则$m\geq 1$,因$-x\leq x\sin\frac{1}{x}\leq x$,夹逼得$f(x)\to 0(x\to 0)$
	\item $m\geq 2,f(x)$在$x=0$可导
	\[\lim_{\Delta x\to 0}\frac{f(0+\Delta x)-f(0)}{\Delta x}=\lim_{\Delta x\to 0}(\Delta x)^{m-1}\sin\frac{1}{\Delta x}\]
	要使上式极限存在,则$m-1\geq 1\implies m\geq 2$
	\item $m\geq 3,f(x)$在$x=0$连续可导
	\[\lim_{x\to 0}f'(x)=\lim_{x\to 0}(mx^{m-1}\sin\frac{1}{x}-x^{m-2}\cos\frac{1}{x})=0\]
	同$(2)$理,$m-1\geq 1$且$m-2\geq 1\implies m\geq 3$
\end{enumerate}
\end{analysis}
由例\ref{egcondiff}可以看出这几个术语的区别,如当$m=2$时,$f(x)$是病态的(pathological),因其在$x=0$处可导但不连续可导.
\par 下面这个补充的定理来自课本习题,结论十分简单,但相当有用.
\begin{theorem}[达布(Darboux)定理]
$f(x)$在$[a,b]$可导,且$f'(a)\neq f'(b)$,$k$为介于$f'(a)$与$f'(b)$之间的任意实数,则至少存在一点$\xi\in(a,b)$,使得$f'(\xi)=k$
\end{theorem}
\begin{analysis}
法一:\\
由上面的讨论知此题不能直接用连续函数介值定理做,因为题设只给了$f(x)$可导,但其导函数不一定连续,故我们需要另辟蹊径.\\
构造函数$F(x)=f(x)-kx$,因为$F(x)$在$[a,b]$连续,由闭区间连续函数最值定理,在$[a,b]$上必存在$f(x)$的最值点,记$\xi=\arg\max\,f(x)$.若$\xi\in(a,b)$,由费马定理立得$F'(\xi)=0$,即$f'(\xi)=k$.\\
下证$\xi\ne a,b$.\\
显然$F(x)$在$(a,b)$上可导,且
\[F'(a)=f'(a)-k,F'(b)=f'(b)-k\]
由题设知$F'(a),F'(b)$一正一负,不妨设$F'(a)>0,F'(b)<0$,由导数的定义我们有
\[F'(a)=\lim_{x\to a^+}\frac{F(x)-F(a)}{x-a}>0\]
\[F'(b)=\lim_{x\to b^-}\frac{F(x)-F(b)}{x-b}<0\]
再由函数极限的局部保号性,存在$\delta_1$,使得$x\in(a,a+\delta_1)$有
\[\frac{F(x)-F(a)}{x-a}>\frac{F'(a)}{2}>0\]
存在$\delta_2$,使得$x\in(b-\delta_2,b)$有
\[\frac{F(x)-F(b)}{x-b}<\frac{F'(b)}{2}<0\]
则
\[F(a+\frac{\delta_1}{2})>F(a),F(b-\frac{\delta_2}{2})>F(b)\]
故我们找到两个点$\displaystyle a+\frac{\delta_1}{2}$和$\displaystyle b-\frac{\delta_2}{2}$,它们的函数值分别大于$a,b$两点的函数值,因此最大值点不会落在$a,b$上.
\end{analysis}
\begin{analysis}
法二:\\
不妨设$f'(b)<k<f'(a)$(类似于增函数),考虑$k$与$\displaystyle\frac{f(b)-f(a)}{b-a}$的关系,不外乎三种可能
\begin{enumerate}
	\item $k=\displaystyle\frac{f(b)-f(a)}{b-a}$,则直接利用微分中值定理即得证
	\item $k>\displaystyle\frac{f(b)-f(a)}{b-a}$,令
	\[g(x)=\begin{cases}
	f'(a),\,x=a\\
	\displaystyle\frac{f(x)-f(a)}{x-a},a<x\leq b\end{cases}\]
	有$\displaystyle g(a)=f'(a)>k>\frac{f(b)-f(a)}{b-a}=g(b)$(假设),由于$g(x)$为连续函数,则由介值定理,
	\[\exists x_0\in(a,b)\:s.t.\:g(x_0)=\frac{f(x_0)-f(a)}{x_0-a}=k\]
	进而由微分中值定理,
	\[\exists \xi\in(a,x_0)\:s.t.\:f'(\xi)=\frac{f(x_0)-f(a)}{x_0-a}=k\]
	\item $k<\displaystyle\frac{f(b)-f(a)}{b-a}$,令
	\[g(x)=\begin{cases}
	f'(b),\,x=b\\
	\displaystyle\frac{f(b)-f(x)}{b-x},a\leq x<b\end{cases}\]
	同理可证.\\
	注:这种方法的思想即先找到一条割线,再通过拉格朗日中值定理找切线.
\end{enumerate}
\end{analysis}
我们把导函数$f'(x)$可以取$f'(a)$与$f'(b)$之间任意值的这种性质称为介值性(intermediate value property). 由这个证明可以看出,达布定理只需要可导,而不需要导函数连续,这是比连续函数介值定理更强的性质. 
\begin{corollary}
\label{coro1}
若$f(x)$在$[a,b]$上可微,则$f'(x)$在$[a,b]$上不可能有第一类间断点
\end{corollary}
\begin{analysis}
若不然,$\exists c\in[a,b]$是$f'(x)$的一个第一类间断点.\\
若$c\in(a,b)$,不妨设$f'(c^-)<f'(c^+)$,任取$r,s\in\mathbb{R}$使得$f'(c^-)<r<s<f'(c^+)$且$f'(c)\ne [r,s]$. (通过$sgn(x)$的图像脑补这种构造)
由函数极限的局部保序性,$\exists\delta>0\,s.t.\,(c-\delta,c+\delta)\subset(a,b)$,且当$x\in(c-\delta,c)$时,$f'(x)<r$;当$x\in(c,c+\delta)$时,$f'(x)>s$.\\
于是当$x\in\mathring{U}(c,\delta)$时,$f'(x)\notin[r,s]$,这与达布定理的介值性矛盾.\\
同理可证得$c=a$或$c=b$的情况,进而$f'(x)$在$[a,b]$上不可能有第一类间断点.\\
注:本题也可用拉格朗日中值定理来求解.
\end{analysis}
\begin{theorem}
\label{mono_and_conti}
$f(x)$在$(a,b)$上可导,且$f'(x)$单调,证明$f'(x)$在$(a,b)$连续
\end{theorem}
\begin{analysis}
由例\ref{conti1}知,若$f'(x)$有间断点,则必为第一类间断点;但推论\ref{coro1}又告诉我们$f'(x)$不可能有第一类间断点,两者相矛盾. 故$f'(x)$在$(a,b)$上没有间断点,即$f'(x)$在$(a,b)$连续
\end{analysis}
同时推论\ref{coro1}也说明了:存在这样的函数,如$sgn(x)$,在区间上是黎曼可积的,但不存在原函数.
\begin{corollary}
$f(x)$在$(a,b)$上可微,若在$[a,b]$上$f'(x)\ne 0$,则$f'(x)$在$[a,b]$上保持同号(恒正或恒负)
\end{corollary}
下面的练习是中值定理和达布定理的综合应用,请读者自行尝试.(当然也有其他方法)
\begin{exercise}
$f(x)$在$[0,1]$上连续,在$(0,1)$内可导,且$f(0)=f(1)=0$,证明$\forall x_0\in(0,1),\exists\xi\in(0,1)\,s.t.\,f'(\xi)=f(x_0)$
\end{exercise}

关于一般题目中值定理的运用请见例\ref{egrolle1},常规思路即构造函数(具体怎么构造见例\ref{constructfun}),目的是:
\begin{enumerate}
	\item 补充定义保证函数在整个区间上\textbf{连续}
	\item 通过代数变换使趋于无限的极限过程变为在\textbf{有限}区间上讨论
	\item 若用罗尔定理函数能够保证端点值相等
\end{enumerate}
\begin{example}
\label{egrolle1}
设$f(x)$在$(a,+\infty)$上可导,且$\displaystyle\lim_{x\to a^+}f(x)=\lim_{x\to +\infty}f(x)=A$,证明:存在$\xi\in(a,+\infty)$,使得$f'(\xi)=0$
\end{example}
\begin{analysis}
令$b>a$,构造
\[F(t)=\begin{cases}
\displaystyle f(\frac{a-b}{t-b}a)\quad t\in(a,b)\\
A\quad t=a,b
\end{cases}\qquad\mbox{补充定义}\]
则易知$F(x)$在$[a,b]$上连续,在$(a,b)$可导,且$F(a)=F(b)=A$\\
由罗尔定理,$\exists\xi_0\in(a,b)$使得
\[F'(\xi)=-\frac{(a-b)a}{(\xi_0-b)^2}f'(\frac{a-b}{\xi_0-b}a)=0\]
$\because b>a,\xi_0\in(a,b)\qquad\displaystyle\therefore f'(\frac{a-b}{\xi_0-b}a)=0$\\
故取$\displaystyle\xi=\frac{a-b}{\xi_0-b}a\in(a,+\infty)$,即有$f'(\xi)=0$
\end{analysis}
\begin{example}
\label{constructfun}
$f(x)$在$(-\infty,+\infty)$连续,且$\displaystyle\lim_{x\to\pm \infty}f(x)=+\infty$,证明$f(x)$在$(-\infty,+\infty)$上取到它的最小值
\end{example}
\begin{analysis}
根据构造函数的目的,我们希望找到这样一个函数$x\to +\infty(t\to a),x\to -\infty(t\to b)$,那$u(t)=\tan t$不失为一个好函数.\\
同样我们希望$v(t)\to c(f(x)\to +\infty)$,不假思索地,用$v(t)=\arctan t$即可满足条件.\\
则我们构造出
\[g(t)=\begin{cases}
\displaystyle\arctan f(\tan t),\,t\in(-\frac{\pi}{2},\frac{\pi}{2})\\
\displaystyle\frac{\pi}{2},\,t=-\frac{\pi}{2},\frac{\pi}{2}
\end{cases}\]
虽然函数长得比较难看,但它是符合题设条件的,即$\displaystyle g(t)\to \frac{\pi}{2},f(x)\to+\infty(t\to \frac{\pm\pi}{2},x\to \pm\infty)$.\\
又$g(t)$在$\displaystyle[-\frac{\pi}{2},\frac{\pi}{2}]$上连续,由连续函数最值定理知存在最小值,又由极限的保号性,知最小值一定不会在端点处取到,进而得证.\\
注:连续不一定可导,故本题不可用罗尔定理
\end{analysis}
\begin{example}
$f(x)$在$[a,+\infty)$有界,$f'(x)$存在,证明$A=\displaystyle\lim_{x\to+\infty}f'(x)=0$
\end{example}
\begin{analysis}
使用反证法,不妨设$A>0$,则由极限的保号性知$\displaystyle\exists x_0,x>x_0:f'(x)>\frac{A}{2}>0$\\
对区间$[x_0,x]$用拉格朗日中值定理有$\exists\xi\in(x_0,x):\displaystyle f'(\xi)=\frac{f(x)-f(x_0)}{x-x_0}$,进而$\displaystyle f(x)=f(x_0)+f'(\xi)(x-x_0)>f(x_0)+\frac{A}{2}(x-x_0)$,故$\displaystyle\lim_{x\to+\infty}f(x)=+\infty$,与$f(x)$有界矛盾.
\end{analysis}


\subsection{微分方程}
通过分离变量法求解一阶微分方程,注意要将微分算子看成是可以运算的(乘除).
\begin{example}
单摆的近似运动方程.
\end{example}
\begin{analysis}
由牛顿第二定律有
\[m\dfrac{\diff^2 s}{\diff t^2}=-mg\sin\theta\]
其中,$s$为扫过的弧长,因圆周角$\disp\theta=\dfrac{s}{l}$,故
\[\dfrac{\diff^2 s}{\diff t^2}=-g\sin\dfrac{s}{l}\]
该方程比较难解,我们只考虑当$\theta\to 0$时,上式变为
\[\dfrac{\diff^2 s}{\diff t^2}=-g\dfrac{s}{l}\]
令$\dfrac{\diff s}{\diff t}=p(s)$,则
\[\dfrac{\diff^2 s}{\diff t^2}=\dfrac{\diff p}{\diff t}=\dfrac{\diff p}{\diff s}\dfrac{\diff s}{\diff t}=\dfrac{\diff p}{\diff s}p\]
故可将方程变为一阶方程
\[p\dfrac{\diff p}{\diff s}=-\dfrac{g}{l}s\]
分离变量得
\[p\diff p=-\dfrac{g}{l}s\diff s\]
解得
\[p^2=-\dfrac{g}{l}s^2+C_1\]
最大偏离时弧长为$s_0$,此时速度为$0$,可得$C_1$,于是
\[\lrp{\dfrac{\diff^2 s}{\diff t^2}}^2=\dfrac{g}{l}(s_0^2-s^2)\]
开方,分离变量积分得
\[\arcsin\dfrac{s}{s_0}=\sqrt{\dfrac{g}{l}}t+C_2\]
假定开始时,摆处于铅垂线上,则
\[s=s_0\sin\sqrt{\dfrac{g}{l}}t\]
即为周期$T=2\pi\sqrt{\dfrac{l}{g}}$的简谐运动.
\end{analysis}